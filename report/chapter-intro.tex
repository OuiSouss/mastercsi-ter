\chapter*{Introduction}
\addcontentsline{toc}{chapter}{Introduction}

Ce projet consiste à s'interresser aux attaques HTTPS, un protocole majoritairement utilisé de nos jours, créé en 1994. HTTPS est l'acronyme pour protocole de transfert hypertexte sécurisé. Il utilise le protocole HTTP, protocole de transfert hypertexte, ainsi qu'une couche de sécurisation. Cette couche sert au chiffrement du flux de données grâce à des protocoles tel que SSL et TLS.

Le HTTPS permet à un visiteur de vérifier l'authenticité du site web qu'il visite grâce à un certificat d'authentification émis par une autorité tierce, réputée et de confiance. Théoriquement, il garantit la confidentialité et l'intégrité des données envoyées par l'utilisateur et reçues par le serveur. Dans un premier temps, les sites internet ont appliqué ce protocole seulement à des parties de leurs sites, il était appliqué après le POST d'un formulaire par example. De nos jours, la majorité des sites proposants un accès restreint utilisent le HTTPS sur tout le site.

%%% Parler de SSL et TLS %%%


%%% Présentation du projet %%%
% But : %

%% 1 - état de l'art pour présenter les attaques qui ont put être faites %%

%% 2 - Réalisation Preuve de concept %%
% utilisation qemunet environnement %
% SSLStrip une première attaque réalisé dans 3 versions %
% HTTPS-Interception %

% CECI représente notre plan comme ça %
